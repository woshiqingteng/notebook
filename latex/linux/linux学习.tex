\documentclass{article}

\usepackage{geometry}
\geometry{a4paper,left=1.25in,right=1.25in,top=1in,bottom=1in}

\usepackage{fontspec}
\setmainfont{Sarasa Gothic SC}

\usepackage{hyperref}
\usepackage{xcolor}

\usepackage{tikz}

\renewcommand{\contentsname}{\centering\huge\textbf{目录}}
\renewcommand{\tablename}{表格}
\renewcommand{\figurename}{图形}
\setlength{\parindent}{0ex}
\setlength{\abovecaptionskip}{2ex}
\setlength{\belowcaptionskip}{2ex}

\title{\Huge\textcolor{orange!70}{Linux学习笔记}}
\author{}
\date{}

\begin{document}

\maketitle
\tableofcontents
\thispagestyle{empty}

\newpage
\setcounter{page}{1}
\section{linux基本概念}

\subsection{路径:描述文件位置的方法}

\begin{enumerate}
    \item 绝对路径:从根目录(/)开始,描述文件在磁盘位置的方法。
    \item 相对路径:从当前位置出发,描述到达目标文件的位置方法。
\end{enumerate}

\begin{table}[htb]
\centering
\caption{\textbf{特殊目录}}
\begin{tabular}{c|c}
\hline
\textbf{特殊目录}&\textbf{描述}\\ 
\hline
/&根目录\\
\~{}&当前用户目录\\
.&当前目录\\
\verb|..|&上一级目录\\
\verb|../..|&上上级目录\\
\hline
\end{tabular}
\end{table}

\section{目录操作命令}
\subsection{目录切换:cd}
基本用法:cd directory
\begin{enumerate}
    \item cd / 切换到根目录
    \item cd /user 切换到根目录下的user目录
    \item cd ../ 切换到上一级目录(此处的/可以省略)
    \item cd \~{} 切换到当前用户目录的家目录
    \item cd - 切换到上次访问的目录
\end{enumerate}
\subsection{目录查看:ls}
基本用法:直接在当前目录下输入ls或ls+目录目标的正确路径
\begin{enumerate}
    \item ls 查看当前目录下的目录和文件
    \item ls -a 查看当前目录下包含隐藏文件在内的目录和文件
    \item ls -l 列表查看当前目录下所有文件和信息
    \item ls /dir 查看指定目录下的所有目录和文件
\end{enumerate}

\subsection{创建目录:mkdir}
基本用法:mkdir 目录名
\begin{enumerate}
    \item mkdir text 在当前目录下创建名为text的目录
    \item mkdir */text 在指定目录下创建名为text的目录
\end{enumerate}

\subsection{删除目录:rm}
基本用法:rm -r 目录名 或 rm -rf 目录名
\begin{enumerate}
    \item rm -r text 递归删除当前目录下的text(有询问)
    \item rm -rf text 递归删除当前目录下的text(无询问)
\end{enumerate}

\subsection{修改目录:mv和cp}
基本用法:mv 目录1 目录2 cp -r 目录1 目录2
\begin{enumerate}
    \item mv 当前目录 新目录 在两个目录上一级目录相同的情况下,可以对目录实现重命名
    \item mv 目录1 目录1要移动的位置 将目录剪切到其他位置
    \item cp -r 目录1 目录1要拷贝到的位置 将目录复制到其他位置
\end{enumerate}

\section{文件操作命令}
\subsection{新建文件:touch}
基本用法:touch 文件名\\
查看文件内容:cat (查看最后一屏内容)\\
more (百分比显示,按回车键增加百分比)

\subsection{修改文件权限:chmod octal file}
\begin{enumerate}
    \item chmod 777 为所有用户添加读写执行权限
    \item chmod 755 为所有者添加rwx权限,为其他用户添加rx权限
    \item -R:对当前目录下的所有文件与子目录进行相同的权限变更()即以递归的方式逐个变更\\
    chmod -R 755 directory
\end{enumerate}

\subsection{压缩/解压文件}
\begin{table}[htb]
    \centering
    \caption{\textbf{压缩/解压文件}}
    \begin{tabular}{l|l}
        \hline
        \hspace{3.5em}\textbf{命令}&\hspace{5em}\textbf{作用}\\
        \hline
        tar cf file.tar file&创建包含file的tar文件file.tar\\
        tar xf file.tar&从file.tar中提取被打包的文件\\
        tar czf file.tar.gz file&使用Gzip压缩创建压缩文件\\
        tar xzf file.tar.gz&使用Gzip提取压缩文件\\
        tar cjf file.tar.bz2 file&使用Bzip2压缩创建压缩文件\\
        tar xjf file.tar.bz2&使用Bzip2提取压缩文件\\
        \hline
    \end{tabular}
\end{table}
\begin{itemize}
    \item -C <目的目录>或--directory=<目的目录>:切换到指定的目录
    \item -z或--gzip或ungzip:通过gzip指令备份文件
    \item -c或--create:建立新的备份文件
    \item -x或--extract或--get:从备份文件中还原文件
    \item -v或--verbose:显示指令执行过程
    \item -f<备份文件>或--file=<备份文件>:指定备份文件
\end{itemize}

\section{vim编辑器}

\begin{figure}[htb]
    \centering
    \begin{tikzpicture}
        \draw (0,0) rectangle (2,1);
        \draw (4,0) rectangle (6.5,1);
        \draw (2,3) rectangle (4,4);
        \draw[->] (1,4.5) -- node[above] {进入} (2.5,4.5) -- (2.5,4);
        \draw[->] (3.5,4) -- (3.5,4.5) -- node[above] {退出} (5,4.5);
        \node (D) at (1.5,4.2) {vi filename};
        \node (E) at (4.2,4.2) {输入:wq};
        \draw[<-] (0.8,1) -- node[above left] {输入i a o} (2.2,3);
        \draw[->] (1.2,1) -- node[below right] {ESC键} (2.6,3);
        \draw[<-] (4.8,1) -- node[below left] {:} (3.2,3);
        \draw[->] (5.2,1) -- node[above right] {命令以回车结束运行} (3.6,3);
        \node[red] (A) at (1,0.5) {输入模式};
        \node[red] (B) at (5.2,0.5) {底线命令模式};
        \node[red] (C) at (3,3.5) {命令模式};
    \end{tikzpicture}
    \caption{vim/vi工作模式}
\end{figure}

\begin{itemize}
    \item 命令模式:Command mode
    \item 输入模式:Insert mode
    \item 底线命令模式:Last line mode
\end{itemize}

\begin{table}[htb]
    \centering
    \caption{文本编辑命令}
    \begin{tabular}{c|c}
        \hline
        \textbf{命令(一般模式)}&\textbf{作用}\\
        \hline
        i&从当前光标所在处输入\\
        a&从目前光标所在处的下一个字符处开始输入\\
        0&从目前光标所在处的下一行处输入新的一行\\
        \hline
        \textbf{命令(底行模式)}&\textbf{作用}\\
        \hline
        :w&保存\\
        :q&退出\\
        :wq&退出且保存\\
        :q!&强制退出\\
        :w!&强制保存\\
        :上下箭头&历史命令\\
        ZZ(大写)&如果修改过,保存当前文件,然后退出。效果等同于(保存并退出)\\
        \hline
    \end{tabular}
\end{table}

\subsection{vim移动光标}
\begin{itemize}
    \item h 或 向左箭头($\leftarrow$):光标向左移动一个字符
    \item j 或 向下箭头($\downarrow$):光标向下移动一个字符
    \item k 或 向上箭头($\uparrow$):光标向上移动一个字符
    \item l 或 向右箭头($\rightarrow$):光标向右移动一个字符
\end{itemize}

\subsection{vim删除撤销命令}
\begin{itemize}
    \item x:向后删除一个字符(相当于\verb|[del]|按键)
    \item X:向前删除一个字符(相当于\verb|[backspace]|亦即是退格键)
    \item dd:删除光标所在一整行
\end{itemize}

\subsection{vim查找和替换}
\begin{itemize}
    \item /string:向下寻找名为string的字符串
    \item ?string:向上寻找名为string的字符串
    \item :n1,n2s/word1/word2/g:在第n1行与n2行之间寻找word1这个字符串,\\并将该字符串取代为word2
    \item :1,\$s/word1/word2/g或:\%s/word1/word2/g:在第一行到最后一行寻找\\word1这个字符串,并将该字符串取代为word2
\end{itemize}


\end{document}