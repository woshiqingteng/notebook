
%3.1章节和目录
%3.1.1章节标题
%\chapter{title} \section{title} \subsection{title}
%\subsubsection{title} \paragraph{title} \subparagraph{title}

%\chapter{}只在repor和book文档中有定义
%\part将文章分为较大的分块,不影响\chapter或\section编号

%\section[short title]{title}
%标题中使用title参数,在目录和页眉页脚中使用short title参数
%\section*{title}
%标题不带编号,也不生成目录项和页眉页脚
%较低层次\paragraph和\subparagraph即使不用带星号的变体,
%生成的标题默认不带编号

%除\part外
%article文档带编号的层级\section / \subsection / \subsubsection 三级
%report/book文档带编号的层级\chapter / \section /\subsubsection 三级

%3.1.2目录
%/tableofcontents
%使用\chapter*或\section*不生成目录项,手动生成该章节目录项
%在标题命令后使用:
%/addcontentsline{toc}{level}{title}
%level为章节层次,chapter或section
%title目录项的章节标题

%3.1.3文档结构的划分
%所有标准文档提供\appendix命令将正文和附录分开
%使用\appendix后,最高一级章节使用拉丁字母编号,从A开始

%book类文档提供前言、正文、后记
%\frontmatter前言,页码使用小写罗马数字;其后\chapter不编号
%\mainmatter正文,页码使用阿拉伯数字,从1开始计数;其后章节编号正常
%\backmatter后记,页码格式不变,继续正常计数;其后\chapter不编号

%源码3.1:book文档类的文档示例
\documentclass{ctexbook}
%导言区,加在宏包和各项设置,包括参考文献、索引等

\title{Test title}
\author{ Maru\thanks{E-mail:*****@***.com}
\and Ted\thanks{Corresponding author}
\and Louis}
\date{today}
%\title{}\author{}必须,不用报错,\date{}可选
%\date默认使用\today

\usepackage{array,tabularx,booktabs,multirow,makecell,graphicx,subcaption}

\graphicspath{{figures/}}

\usepackage{makeidx}%调用makeidx宏包,用来处理索引
\makeindex%开启索引的收集
\bibliographystyle{plain}%指定参考文献样式为plain
\begin{document}
    
\frontmatter%前言
\maketitle%标题页article默认不单独成页,report和book默认单独成页
\begin{titlepage}
    \dots
\end{titlepage}
\chapter{preface}
\include{preface}%前言章节preface.tex
\tableofcontents

\mainmatter%正文
\chapter{chapter1}
\include{chapter1}%第一章chapter1.tex
\chapter{chapter2}
\include{chapter2}%第二章chapter2.tex
Some excellent \index{study} books, for example, \cite{Alice13}
...

%使用 \appendix 后,最高一级章节改为使用拉丁字母编号,从 A 开始
\appendix%附录
\chapter{sin}
\include{appendixA}%附录A appendix.tex
... 

\backmatter%后记部分
\chapter{}
\include{epilogue}%后记epilogue.tex
\bibliography{books}%利用BibTeX工具从数据库文件books.bib生成参考文献
\printindex%利用makeindex工具生成索引


A reference to this subsection \label{sec:this}
look like :
``see section~\ref{sec:this} on page~\pageref{sec:this}.''

“天地玄黄,宇宙洪荒。日月盈昃,辰宿列张。”\footnote{出自《千字文》。}

\begin{tabular}{l}
\hline
“天地玄黄,宇宙洪荒。日月盈昃,辰宿列张。”\footnotemark[5] \\
\hline
\end{tabular}
\footnotetext[5]{表格里的名句出自《千字文》}


\marginpar{\footnotesize 边注较窄,不要写过多文字,
最好设置较小的字号}


\begin{enumerate}
    \item An item.
    \begin{enumerate}
        \item A nested item.\label{itref}
        \item[*] A starred item.
    \end{enumerate}
    \item Reference(\ref{itref}).
\end{enumerate}

\begin{itemize}
    \item An item.
    \begin{itemize}
        \item A nested item.
        \item[+] A `plus' item.
        \item Another item.
    \end{itemize}
    \item Go back to upper level.
\end{itemize}

\begin{description}
    \item[Enumerate] Numbered list.
    \item[Itemize] Non-numbered list. 
\end{description}

\renewcommand{\labelitemi}{\ddag}
\renewcommand{\labelitemii}{\dag}
\begin{itemize}
    \item First item
    \begin{itemize}
        \item Subitem
        \item Subitem
    \end{itemize}
    \item Second item
\end{itemize}

\renewcommand{\labelenumi}{\Alph{enumi}>}
\begin{enumerate}
    \item First item
    \item Second item
\end{enumerate}

\begin{center}
    Centered text using a \verb|center| environment.
\end{center}
\begin{flushleft}
    Left-aligned text using a \verb|flushleft| environment.
\end{flushleft}
\begin{flushright}
    Right-aligned text using a \verb|flushright| environment.
\end{flushright}

%\centering
%Centered text paragraph.

%\raggedright
%Left-aligned text paragraph.

%\raggedleft
%Rigrh-aligned text paragraph.


Francie Bacon says:
\begin{quote}
    Knowledge is power.
\end{quote}

《木兰诗》:
\begin{quotation}
万里赴戎机,关山度若飞。
朔气传金柝,寒光照铁衣。
将军百战死,壮士十年归。

归来见天子,天子坐明堂。
策勋十二转,赏赐百千强。⋯⋯
\end{quotation}

Rabindranath Tagore's short poem:
\begin{verse}
Beauty is truth's smile
when she beholds her own face in a perfect mirror.
\end{verse}

\begin{verbatim}
#include<iostream>
int main()
{
    std::cout<<"Hello,world!"
             << std::endl;
    return 0;
}
\end{verbatim}

\begin{verbatim*}
for (int i=0; i<4; ++i)
  printf("Number %d\n",i);
\end{verbatim*}

\verb|\LaTeX|\\
\verb+(a || b)+ \verb*+(a || b)+

\begin{tabular}{|c|}
    center-\\aligned \\
\end{tabular},
\begin{tabular}[t]{|c|}
    top-\\aligned \\
\end{tabular},
\begin{tabular}[b]{|c|}
    bottom-\\aligned \\
\end{tabular} tabulars.

\begin{tabular}{lcr|p{6em}}
    \hline
    left & center & right
         & par box with fixed width\\
    L    & C      & R   & P\\
    \hline
\end{tabular}

\begin{tabular}{@{} r@{:}lr @{}}
    \hline
    1  & 1 & one \\
    11 & 3 & eleven \\
    \hline
\end{tabular}

% \usepackage{array}
%array 宏包提供了辅助格式 > 和 <,
%用于给列格式前后加上修饰命令
\begin{tabular}{>{\itshape}r<{*}l}
    \hline
    italic & normal \\
    column & columc \\
    \hline
\end{tabular}

% \usepackage{array}
\begin{tabular}{>{\centering\arraybackslash}p{9em}}
    \hline
    Some center-aligned long text. \\
    \hline
\end{tabular}

% \usepackage{array}
\newcommand\txt{a b c d e f g h i}
\begin{tabular}{cp{2em}m{2em}b{2em}}
    \hline
    pos & \txt & \txt & \txt \\
    \hline
\end{tabular}

\begin{tabular*}{14em}%
    {@{\extracolsep{\fill}}|c|c|c|c|}
    \hline
    A & B & C & D \\ \hline
    a & b & c & d \\ \hline
\end{tabular*}

% \usepackage{array,tabularx}
\begin{tabularx}{14em}%
{|*{4}{>{\centering\arraybackslash}X|}}
\hline
A & B & C & D \\ \hline
a & b & c & d \\ \hline
\end{tabularx}

\begin{tabular}{|c|c|c|}
    \hline
    4 & 9 & 2 \\ \cline{2-3}
    3 & 5 & 7 \\ \cline{1-1}
    8 & 1 & 6 \\ \hline
\end{tabular}

%\begin{tabular}{cccc}
%    \toprule
%    & \multicolumn{3}{c}{Numbers} \\
%    \cmidrule{2-4}
%    & 1 & 2 & 3 \\
%    \midrule
%    Alphabet & A & B  & C \\
%    Roman    & I & II & III \\
%   \bottomrule
%\end{tabular}


\begin{tabular}{|c|c|c|}
    \hline
    1 & 2 & Center \\ \hline
    \multicolumn{2}{|c|}{3} &
    \multicolumn{1}{r|}{Right} \\ \hline
    4 & \multicolumn{2}{c|}{C} \\ \hline
\end{tabular}

\begin{tabular}{ccc}
    \hline
    \multirow{2}{*}{Item} &
    \multicolumn{2}{c}{Value} \\
    \cline{2-3}
    & First & Second \\ \hline
  A & 1     & 2 \\ \hline 
\end{tabular}

\begin{tabular}{|c|c|c|}
    \hline
    a & b & c \\ \hline
    a & \multicolumn{1}{@{}c@{}|}
    {\begin{tabular}{c|c}
    e & f \\ \hline
    e & f \\ 
    \end{tabular}}
          & c \\ \hline
    a & b & c \\ \hline
\end{tabular}

% \usepackage{makecell}
\begin{tabular}{|c|c|}
    \hline
    a & \makecell{d1 \\ d2} \\
    \hline
    b & c \\
    \hline
\end{tabular}

\renewcommand\arraystretch{1.8}
\begin{tabular}{|c|}
    \hline
    Really loose \\ \hline
    tabular rows. \\ \hline
\end{tabular}

\begin{tabular}{c}
    \hline
    Head lines \\[6pt]
    tabular lines \\
    tabular lines \\ \hline
\end{tabular}

%\includegraphics[scale=0.2]{1.png}
%|\mbox{Test some words.}|\\
%|\makebox[10em][l]{Test some words.}|\\
%|\makebox[10em][r]{Test some words.}|\\
%分散对齐方式强行拉开单词的间距,
%往往会报 Underfull \hbox 的警告
%|\makebox[10em][s]{Test some words.}|

\fbox{Test some words.}\\
\framebox[10em][r]{Test some words.}

\framebox[10em][r]{Test box}\\[1ex]
\setlength{\fboxrule}{1.6pt}
\setlength{\fboxsep}{1em}
\framebox[10em][r]{Test box}

三字经:\parbox[t]{3em}%
{人之初 性本善 性相近 习相远}
\quad
千字文:
\begin{minipage}[b][8ex]{4em}
天地玄黄 宇宙洪荒
\end{minipage}

\fbox{\begin{minipage}{15em}%
    这是一个垂直盒子的测试。
    \footnote{脚注来自minipage。}
\end{minipage}}

black \rule{12pt}{4pt} box.

Upper \rule[4pt]{6pt}{8pt} and 
lower \rule[-4pt]{6pt}{8pt} box.

A  \rule[-.4pt]{3em}{.4pt} line.

%\begin{figure}[htbp]
%    \centering
%    \includegraphics[width=3cm]{1.png}
%    \qquad
%    \includegraphics[width=3cm]{2.png}\\
%    \includegraphics[width=3cm]{3.jpg}
%    \caption{并排}
%\end{figure}

\begin{figure}[htbp]
    \centering
    \begin{minipage}{15em}
        \centering
        \includegraphics[width=3cm]{1.png}
        \caption{图1}
    \end{minipage}
    \qquad
    \begin{minipage}{15em}
        \centering
        \includegraphics[width=3cm]{2.png}
        \caption{图2}
    \end{minipage}
\end{figure}

\begin{figure}[htbp]
    \centering
    \begin{subfigure}{15em}
        \centering
        \includegraphics[width=3cm]{1.png}
        \caption{图一}
    \end{subfigure}
    \qquad
    \begin{subfigure}{15em}
        \centering
        \includegraphics[width=3cm]{2.png}
        \caption{图二}
    \end{subfigure}
\end{figure}



\end{document}



%3.2标题页
%titlepage 环境,生成不带页眉页脚的一页
%利用 titlepage 环境重新定义 \maketitle
%\renewcommand{\maketitle}{\begin{titlepage}
%... % 用户自定义命令
%\end{titlepage}}

%3.3交叉引用
%\label{} 
%\ref{} \pageref{}c生成交叉引用的编号和页码

%\label命令使用位置
%章节标题:\section等之后紧接着使用
%行间公式:单行公式在公式内任意位置使用;
%多行公式在每一行公式的任意位置使用
%有序列表:在enumerate每个\item之后、下一个\item之前任意位置使用
%图表标题:\caption之后紧接着使用
%定理环境:定理环境内部任意位置使用

%再使用不记编号的命令(\section*、\caption*,带可选参数的\item等)
%不要使用\label命令,否则生成引用不正确


%3.4脚注和边注
%\footnote{footnote}

%marginpar[left-margin]{right-margin}
%在边栏位置形成边注
%只给定right-margin,边注在奇偶页文字相同;
%同时给定left-margin,偶数页使用left-margin文字

%3.5特殊环境
%3.5.1列表
%有序列表和无序列表环境enumerate和itemize
%用\item标明每个列表项,enumerate自动对列表项编号
%\begin{enumerate}
%    \item 
%\end{enumerate}
%\item可带可选参数,将有序列表或无序列表符号替换自定义符号
%列表可以嵌套使用,最多嵌套四层

%关键字环境description
%\item后可选参数写关键词,粗体表示,一般必填
%\begin{description}
%    \item[itemtitle] ... 
%\end{description}

%各级无序列表由\labelitemi到\labelitemiv定义

%有序列表由\labelenumi到\

%3.5.2对齐环境
%\begin{center}...\end{center}居中
%\begin{flushleft}...\end{flushleft}左对齐
%\begin{flushright}...\end{flushright}右对齐

%除此之外,还可以使用以下命令改变文字对齐方式
%\centering \raggedright \raggedleft

%\flushleft 或 raggedright不严格用法

%center等环境在上下文产生额外间距,
%\centering等命令不产生,只改变对齐方式
%在浮动体环境table和figure内实现居中对齐,用\centering即可
%没必要再用center环境

%3.5.3引用环境
%quote引用较短文字,首行不缩进
%quotation引用若干段文字,首行缩进。
%引用环境较一般文字有额外的左右缩进

%verse用于排版诗歌,与quotation相反,verse首行悬挂缩进

%3.5.4摘要环境
%摘要环境abstract默认在标准文档article和report可用
%一般紧跟\maketitle之后介绍文档的摘要
%如果文档类制定了titlepage选项,则单独成页;
%反之,单栏排版时相当于居中小标题加一个quotation环境
%双栏排版时相当于、section*定义的一节

%3.5.5代码环境
%verbatim
%等宽字体排版代码,回车和空格也分别起到换行和空位的作用;
%带星号的版本更进一步将空格显示成“␣”

%排版简短代码或关键字,使用\verb命令:
%\verb<delim><code><delim>
%<delim>标明代码的分界位置,前后一致,
%除字母、空格或星号外,可任意选择使得不与代码本身冲突
%习惯上使用|符号

%\verb命令对符号处理比较复杂,一般不能用在其他命令参数里
%否则多半会出错

%verbatim、fancyvrb、listings宏包

%3.6表格
%排版表格最基本的 tabular 环境用法为:
%\begin{tabular}[⟨align⟩]{⟨column-spec⟩}
%⟨item1⟩ & ⟨item2⟩ & … \\
%\hline
%⟨item1⟩ & ⟨item2⟩ & … \\
%\end{tabular}
%其中 ⟨column-spec⟩ 是列格式标记,
%& 用来分隔单元格;\\ 用来换行;
%\hline 用来在行与行之间绘制横线。
%可选参数 ⟨align⟩ 控制垂直对齐:
%t 和 b 分别表示按表格顶部、底部对齐,
%其他参数或省略不写(默认)表示居中对齐。

%3.6.1列格式
%表格中每行的单元格数目不能多于列格式里 l/c/r/p 的总数(可以少于这个总数),否则出错。
%@格式可在单元格前后插入任意的文本,但同时它也消除了单元格前后额外添加的间距。
%@格式可以适当使用以充当“竖线”。特别地,@{} 可直接用来消除单元格前后的间距

%格式参数重复的写法 *{⟨n⟩}{⟨column-spec⟩},比如以下两种写法是等效的:
%\begin{tabular}{|c|c|c|c|c|p{4em}|p{4em}|}
%\begin{tabular}{|*{5}{c|}*{2}{p{4em}|}}

%辅助格式甚至支持插入\centering 等命令改变p列格式的对齐方式,
%一般还要加额外的命令\arraybackslash 以免出错

%3.6.2列宽

%3.6.3横线

%booktabs宏包

%3.6.4合并单元格
%横向合并单元格
%\multicolumn{⟨n⟩}{⟨column-spec⟩}{⟨item⟩}
%⟨n⟩ 为要合并的列数,
%⟨column-spec⟩ 为合并单元格后的列格式,
%只允许出现一个 l/c/r 或p 格式。
%如果合并前的单元格前后带表格线 |,
%合并后的列格式也要带 | 以使得表格的竖线一致

%纵向合并单元格
% multirow 宏包提供的 \multirow 命令:
%\multirow{⟨n⟩}{⟨width⟩}{⟨item⟩}
%⟨width⟩ 为合并后单元格的宽度,可以填 * 以使用自然宽度。

%3.6.5嵌套表格
%makecell宏包

%3.6.6行距控制
%\arraystretch

%3.7图片
%调用graphicx宏包
%\includegraphics[⟨options⟩]{⟨filename⟩
% graphicx 宏包还提供了\graphicspath 命令,
%用于声明一个或多个图片文件存放的目录,
%使用这些目录里的图片时可不用写路径:

% 假设主要的图片放在 figures 子目录下,
%标志放在 logo 子目录下
%\graphicspath{{figures/}{logo/}}

%3.8盒子
%生成水平盒子的命令如下:
%\mbox{…}
%\makebox[⟨width⟩][⟨align⟩]{…}

%3.8.2带框的水平盒子
%\fbox{…}
%\framebox[⟨width⟩][⟨align⟩]{…}

%3.8.3垂直盒子
%\parbox[⟨align⟩][⟨height⟩][⟨inner-align⟩]{⟨width⟩}{…}
%\begin{minipage}[⟨align⟩][⟨height⟩][⟨inner-align⟩]{⟨width⟩}
%    …
%\end{minipage}
%⟨inner-align⟩ 接受的参数是顶部 t、底部 b、居中 c 和分散对齐 s

%3.8.4标尺盒子
%\rule 命令用来画一个实心的矩形盒子,
%也可适当调整以用来画线(标尺):
%\rule[⟨raise⟩]{⟨width⟩}{⟨height⟩}

%3.9浮动体
%以 table 环境的用法举例,figure 同理:
%\begin{table}[⟨placement⟩]
%    …
%\end{table}

%双栏排版环境下,LATEX 提供了 table* 和 figure* 环境用来排版跨栏的浮动体
%双栏的 ⟨placement⟩ 参数只能用 tp 两个位置
%float 宏包为浮动体提供了 H 位置参数
%,不与 htbp 及 ! 混用。使用 H 位置参数时,
%会取消浮动机制,将浮动体视为一般的盒子插入当前位置。

%3.9.1浮动体的标题
%\caption{…}
%\listoftables
%\listoffigures

%3.9.2并排和子图表
%subcaption、subfug宏包