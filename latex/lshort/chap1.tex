%!TEX program = xelatex


%1.1
%\documentclass{article}
%\begin{document}
%   ``Hello world!'' from \LaTeX
%\end{document}

%⟨path⟩ 中的多级目录在 Windows 系统上使用反斜线 \ 分隔,
%而在类 Unix 系统上使用正斜线 / 分隔。
%如果 ⟨path⟩ 中带有空格,则需加上引号 "。
%此外,在 Windows 系统上如果要切换到其他分区,
%还需加上 /d 选项,
%例如 cd /d "C:\Program Files (x86)\"。

%1.2
\documentclass{ctexart}
%导言区,常使用\usepackage命令调用宏包
\begin{document}
%正文内容
    “你好,世界!” 来自 \LaTeX{}的问候。
%1.3.1
    Shall we call ourselves \TeX users
    or \TeX{} users?
    \TeX\ users

\end{document}

%此后内容会被忽略

%1.4.1
%\documentclass[option可选参数]{class-name文档名称}
%[文档指定选项,以全局地规定一些排版参数,
%如字号、纸张大小、单双面]

%{LaTeX提供的article/report/book/proc/slides/minimal,
%支持中文排版的ctexart/ctexrep/ctexbook,
%其他功能的一些文档moderncv/beamer}

%\documentclass[11pt,twoside,a4paper]{article}

%1.4.2宏包
%\usepackage[option]{package-name}

%在Windows命令行提示符下或Linux终端输入查询帮助文档
%texdoc <pkg-name>

%一次性调用三个排版表格常用的宏包
%\usepackage{tabularx,makecell,multirow}

%1.5宏包文件
%.sty宏包文件
%.cls文档类文件
%.bibBIBTEX参考文献数据库文件
%.bst参考文献格式模板

%1.6文件的组织方式
%\include{filename}LATEX2020-10-01之后允许添加扩展名
%不在一个目录添加相对或绝对路径
%\include{chapters/file}%相对路径
%\include{/home/Bob/file}%*nix(包含Linux、macOS)绝对路径
%\include{D:/file}%Windows绝对路径,用正斜体

%/include{}读入文件另起一页
%/input{}文件内容插入

%/includeonly{filename1,filename2,...}用于导言区,指定载入某些文件
%载入文件名不要加空格和特殊字符,尽量避免中文名,可能出错

%1.7



