%!TEX program = xelatex

\documentclass[12pt]{ctexrep}

\usepackage{geometry}
\usepackage{multicol}
\geometry{a4paper,left=1.25in,right=1.25in,top=1in,bottom=1in}
%\usepackage{ulem}



\begin{document}

\setlength{\parindent}{0em}

%5.1字体和字号
{\small The small and
\textbf{bold} Romans ruled}
{\Large all of great big 
{\itshape Italy}.}

%5.1.1字体样式
%LATEX 提供了两组修改字体的命令,
%其中诸如 \bfseries 形式的命令将会影响之后所有的字符,
%如果想要让它在局部生效,需要用花括号分组,
%也就是写成 {\bfseries ⟨sometext⟩} 这样的形式;
%对应的 \textbf 形式带一个参数,只改变参数内部的字体,更为常用。
%在公式中,直接使用 \textbf 等命令不会起效,甚至报错。
%LATEX 提供了修改数学字母样式的命令,如 \mathbf 等

%5.1.2字号
He likes {\LARGE large and 
{\small small} letters}.

%LATEX 还提供了一个基础的命令
%\fontsize 用于设定任意大小的字号:
%\fontsize{⟨size⟩}{⟨base line-skip⟩}
%\fontsize 用到两个参数,
%⟨size⟩ 为字号,⟨base line-skip⟩ 为基础行距。
%如果不是在导言区,\fontsize 的设定需要 
%\selectfont 命令才能立即生效

%5.1.3选用字体宏包

%5.1.4字体编码
%切换字体编码要用到 fontenc 宏包:
%\usepackage[T1]{fontenc}

%如果使用xelatex 编译方式,
%并使用 fontspec 宏包调用 ttf 或 otf 格式字体,
%就不要再使用 fontenc 宏包。

%5.1.5使用fontspec宏包更改字体
%xelatex 和 lualatex 命令下支持用户调用字体的宏包是fontspec。
%宏包提供了几个设置全局字体的命令,
%设置 \rmfamily 等对应命令的默认字体

%\setmainfont{⟨font name⟩}[⟨font features⟩]
%\setsansfont{⟨font name⟩}[⟨font features⟩]
%\setmonofont{⟨font name⟩}[⟨font features⟩]

%其中 ⟨font name⟩ 使用字体的文件名(带扩展名)或者字体的英文名称。
%⟨font features⟩ 用来手动配置对应的粗体或斜体,
%比如为 Windows 下的无衬线字体 Arial 配置粗体和斜体
%(通常情况下自动检测并设置对应的粗体和斜体,无需手动指定):
%\setsansfont{Arial}[BoldFont={Arial Bold}, ItalicFont={Arial Italic}]

%需要注意的是,fontspec 宏包会覆盖数学字体设置。
%需要调用一些数学字体宏包时,
%应当在调用 fontspec 宏包时指定 no-math 选项。
%fontspec宏包可能被其它宏包或文档类(如 ctex 文档类)自动调用时,
%则在文档开头的\documentclass 命令里指定 no-math 选项。

%5.1.6在ctex宏包或文档类中更改中文字体
%\setCJKmainfont{⟨font name⟩}[⟨font features⟩]
%\setCJKsansfont{⟨font name⟩}[⟨font features⟩]
%\setCJKmonofont{⟨font name⟩}[⟨font features⟩]

%由于中文字体少有对应的粗体或斜体,
%⟨font features⟩ 里多用其他字体来配置,
%比如在 Windows 中设定基本字体为宋体,
%并设定对应的 BoldFont 为黑体,ItalicFont 为楷体:
%\setCJKmainfont{SimSun}[BoldFont=SimHei, ItalicFont=KaiTi]

%5.1.7使用unicode-math宏包配置Unicode数学字体
%Unicode 数学字体是一类 OpenType 字体,
%包含了 Unicode 字符集中的数学符号部分,
%字体中也设定了数学公式排版所需的一些参数
%在导言区使用 \usepackage{unicode-math} 后,
%使用 \setmathfont 命令即可:
%\setmathfont{⟨font name⟩}[⟨font features⟩]

%5.2文字装饰和强调
An \underline{underlined} text.

%\underline 命令生成下划线的样式不够灵活,
%不同的单词可能生成高低各异的下划线,并且无法换行。
%ulem (重新定义了\emph{})宏包提供了更灵活的解决方案,
%它提供的 \uline 命令能够轻松生成自动换行的下划线:

%An example of \uline{some long and underlined words.}

%\emph 命令,它将文字变为斜体以示强调,
%而如果在已强调的文字中嵌套使用 \emph 命令,
%命令内则使用直立体文字:
Some \emph{emphasized words,
including \emph{double-emphasized}
words}, are shown here.

%5.3段落格式和间距
%5.3.1长度和长度变量

%可伸缩的“弹性长度”,
%如 12pt plus 2pt minus 3pt 表示基础长度为 12pt,
%可以伸展到 14pt,也可以收缩到 9pt。
%也可只定义 plus 或者 minus 的部分,如 0pt plus 5pt。
%长度的数值还可以用长度变量本身或其倍数来表达,
%如 2.5\parindent 等。

%自定义长度变量,需使用如下命令:
%\newlength{\⟨length command⟩}


%长度变量可以用 \setlength 赋值,或用 \addtolength 增加长度:
%\setlength{\⟨length command⟩}{⟨length⟩}
%\addtolength{\⟨length command⟩}{⟨length⟩}

%5.3.2行距
%前文中我们提到过 \fontsize 命令可以为字号设定对应的行距,
%但我们很少那么用。更常用的办法是在导言区使用\linespread 命令。
%\linespread{⟨factor⟩}
%其中 ⟨factor⟩ 作用于基础行距而不是字号。
%缺省的基础行距是 1.2 倍字号大小(参考 \font-size 命令),
%因此使用 \linespread{1.5} 意味着最终行距为 1.8 倍的字号大小。

%如果不是在导言区全局修改,而想要局部地改变某个段落的行距,
%需要用 \selectfont 命令使 \linespread 命令的改动立即生效:

{\linespread{2.0}\selectfont
The baseline skip is set to be 
twice the normal baseline skip.
Pay attention to the \verb|\par|
command at the end. \par }

In comparison, after the 
curly brace has been closed,
everything is back to normal.

%字号的改变是即时生效的,而行距的改变直到文字分段时才生效

{\Large Don't read this!
It is not true.
You can believe me!
Don't read this!
It is not true.
You can believe me!\par}

{\Large This is not true either.
But remember I am a liar.
This is not true either.
But remember I am a liar.}\par 

%5.3.3段落格式
%以下长度分别为段落的左缩进、右缩进和首行缩进:
%\setlength{\leftskip}{⟨length⟩}
%\setlength{\rightskip}{⟨length⟩}
%\setlength{\parindent}{⟨length⟩}

%它们和设置行距的命令一样,在分段时生效。
%控制段落缩进的命令为:
%\indent
%\noindent

%LATEX 默认在段落开始时缩进,
%长度为用上述命令设置的 \parindent。
%如果需要在某一段不缩进,可在段落开头使用 \noindent 命令。
%相反地,\indent 命令强制开启一段首行缩进的段落。
%在段落开头使用多个 \indent 命令可以累加缩进量。

%LATEX 还默认在 \chapter、\section
% 等章节标题命令之后的第一段不缩进
%如果不习惯这种设定,可以调用 indentfirst 宏包,
%令第一段的首行缩进照常
%ctex 宏包和文档类默认按照中文习惯保持标题后第一段的首行缩进

%段落间的垂直间距为\parskip,如设置段落间距在0.8ex到1.5ex变动:
%\setlength{\parskip}{1ex plus 0.5ex minus 0.2ex}

%5.3.4水平间距
%LATEX 默认为将单词之间的“空格”转化为水平间距。
%如果需要在文中手动插入额外的水平间距,可使用 \hspace 命令:

This\hspace{1.5cm}is a space of 1.5cm.

%\hspace 命令生成的水平间距如果位于一行的开头或末尾,
%则有可能因为断行而被舍弃。
%可使用\hspace* 命令代替\hspace 命令
%得到不会因断行而消失的水平间距。

%命令 \stretch{⟨n⟩} 生成一个特殊弹性长度,参数 ⟨n⟩ 为权重。
%它的基础长度为 0pt,但可以无限延伸,直到占满可用的空间。
%如果同一行内出现多个 \stretch{⟨n⟩},
%这一行的所有可用空间将按每个\stretch 命令给定的权重 ⟨n⟩ 进行分配。
%命令 \fill 相当于 \stretch{1}

x\hspace{\stretch{1}}
x\hspace{\stretch{3}}
x\hspace{\fill}x

%在正文中用 \hspace 命令生成水平间距时,往往使用 em 作为单位,
%生成的间距随字号大小而变。
%我们在数学公式中见过 \quad 和 \qquad 命令,
%它们也可以用于文本中,分别相当于
%\hspace{1em} 和 \hspace{2em}:
{\Large big\hspace{1em}y}\\
{\Large big\quad y}\\
nor\hspace{2em}mal\\
nor\qquad mal\\
{\tiny tin\hspace{1em}y}\\
{\tiny tin\quad y}

%5.3.5垂直间距
%在页面中,段落、章节标题、行间公式、列表、浮动体
%等元素之间的间距是 LATEX 预设的。
%比如 \parskip,默认设置为 0pt plus 1pt。
%如果我们想要人为地增加段落之间的垂直间距,
%可以在两个段落之间的位置使用 \vspace命令:

A paragraph.

\vspace{2ex}
Another paragraph.

%\vspace 命令生成的垂直间距在一页的顶端或底端可能被“吞掉”,
%类似 \hspace 在一行的开头和末尾那样。
%对应地,\vspace* 命令产生不会因断页而消失的垂直间距。
%\vspace 也可用\stretch 设置无限延伸的垂直长度。

%在段落内的两行之间增加垂直间距,
%一般通过给断行命令 \\ 加可选参数,
%如 \\[6pt] 或\\*[6pt]。\vspace 也可以在段落内使用,
%区别在于 \vspace 只引入垂直间距而不断行:

Use command \verb|\vspace{12pt}|
to add \vspace{12pt} some spaces
between lines in a paragraph.

Or you can use \verb|\\[12pt]|
to \\[12pt] add vertical space,
but it also breaks the line.

%另外 LATEX 还提供了
%\bigskip, \medskip, \smallskip 来增加预定义长度的垂直间距

\parbox[t]{3em}{TeX\par TeX}
\parbox[t]{3em}{TeX\par\smallskip TeX}
\parbox[t]{3em}{TeX\par\medskip TeX}
\parbox[t]{3em}{TeX\par\bigskip TeX}

%5.4页面和分栏
%5.4.1利用geometry宏包设置页面参数
%既可以调用 geometry 宏包,
%然后用其提供的 \geometry 命令设置页面参数:
%\usepackage{geometry}
%\geometry{⟨geometry-settings⟩}
%也可以直接在宏包选项中设置:
%\usepackage[⟨geometry-settings⟩]{geometry}
%其中 ⟨geometry-settings⟩ 多以 ⟨key⟩=⟨value⟩ 的形式组织。

%符合 Microsoft Word 习惯的页面设定是 
%A4 纸张,上下边距 1 英寸,左右边距 1.25英寸,
%于是我们可以通过如下两种等效的方式之一设定页边距:
%\geometry{a4paper,left=1.25in,right=1.25in,top=1in,bottom=1in}
% or like this:
%\geometry{a4paper,hmargin=1.25in,vmargin=1in}
%又比如,需要设定周围的边距一致为 1.25 英寸,可以用更简单的语法:
%\geometry{margin=1.25in}
%对于书籍等双面文档,习惯上奇数页右边、偶数页左边留出较大的页边距,
%而靠近书脊一侧的奇数页左边、偶数页右边页边距较小。我们可以这样设定:
%\geometry{inner=1in,outer=1.25in}

%5.4.2页面内容的垂直对齐
%以下命令分别令页面在垂直方向向顶部对齐/分散对齐:
%\raggedbottom
%\flushbottom

%5.4.3分栏
%标准文档类的全局选项 onecolumn、twocolumn 
%可控制全文分单栏或双栏排版。
%LATEX 也提供了切换单/双栏排版的命令:
%\onecolumn
%\twocolumn[⟨one-column top material⟩]
%\twocolumn 支持带一个可选参数,
%用于排版双栏之上的一部分单栏内容。

%切换单/双栏排版时总是会另起一页(\clearpage)。
%在双栏模式下使用 \newpage 会换栏而不是换页;
%\clearpage 则能够换页。

%如以下环境将内容分为 3 栏:
\begin{multicols}{3}
    multicol 宏包能够在一页之中切换单栏/多栏,也能处理跨页的分栏,且各栏的高度分布平
衡。但代价是在 multicols 环境中无法正常使用 table 和 figure 等浮动体环境,它会直接让
浮动体丢失。multicols 环境中只能用跨栏的 table* 和 figure* 环境,或者用 float 宏包提供
的 H 参数固定浮动体的位置。
\end{multicols}
%multicol 宏包能够在一页之中切换单栏/多栏,
%也能处理跨页的分栏,且各栏的高度分布平衡。
%但代价是在 multicols 环境中无法正常使用 table 和 figure 等浮动体环境,
%它会直接让浮动体丢失。
%multicols 环境中只能用跨栏的 table* 和 figure* 环境,
%或者用 float 宏包提供的 H 参数固定浮动体的位置。

%5.5页眉和页脚
%LATEX 中提供了命令 \pagestyle 来修改页眉页脚的样式:
%\pagestyle{⟨page-style⟩}
%命令 \thispagestyle 只影响当页的页眉页脚样式:
%\thispagestyle{⟨page-style⟩}
%⟨page-style⟩ 参数为样式的名称,在 LATEX 里预定义了四类样式
%empty/plain/headings/myheadings

%article 文档类,twoside 选项 偶数页为页码和节标题,奇数页为小节标题和页码;
%article 文档类,oneside 选项 页眉为节标题和页码;
%report / book 文档类,twoside 选项 偶数页为页码和章标题,奇数页为节标题和页码;
%report / book 文档类,oneside 选项 页眉为章标题和页码。

%\pagenumbering 命令令我们能够改变页眉页脚中的页码样式:
%\pagenumbering{⟨style⟩}
%⟨style⟩ 为页码样式,默认为 arabic(阿拉伯数字),
%还可修改为 roman(小写罗马数字)、Roman(大写罗马数字)等

%注意使用 \pagenumbering 命令后会将页码重置为 1。
%book 文档类的 \frontmatter 和 \mainmatter 内部就使用了 
%\pagenumbering 命令切换页码样式。

%5.5.2手动更改页眉页脚的内容
%\markright{⟨right-mark⟩}
%\markboth{⟨left-mark⟩}{⟨right-mark⟩}
%在双面排版、headings / myheadings 页眉页脚样式下,
%⟨left-mark⟩ 和 ⟨right-mark⟩
% 的内容分别预期出现在左页(偶数页)和右页(奇数页)。
%事实上 \chapter 和 \section 等章节命令
%内部也使用 \markboth 或者 \markright 生成页眉。

%LATEX 默认将页眉的内容都转为大写字母。
%如果需要保持字母的大小写,可以尝试以下代码
%\renewcommand\chaptermark[1]{%
%\markboth{Chapter \thechapter\quad #1}{}}
%\renewcommand\sectionmark[1]{%
%\markright{\thesection\quad #1}}
%其中 \thechapter、\thesection 等命令为章节计数器的数值
%以上代码适用于 report / book 文档类。
%对于 article 文档类,与两个页眉相关的命令分别为
%\sectionmark和 \subsectionmark。

%5.5.3fancyhdr宏包
%fancyhdr 宏包改善了页眉页脚样式的定义方式,
%允许我们将内容自由安置在页眉和页脚的
%左、中、右三个位置,还为页眉和页脚各加了一条横线。
%fancyhdr 自定义了样式名称 fancy。
%使用 fancyhdr 宏包定义页眉页脚之前,
%通常先用 \pagestyle{fancy} 调用这个样式。
%在 fancyhdr 中定义页眉页脚的命令为:
%\fancyhf[⟨position⟩]{…}
%\fancyhead[⟨position⟩]{…}
%\fancyfoot[⟨position⟩]{…}
%其中 ⟨position⟩ 为 L(左)/C(中)/R(右)以及与 O(奇数页)/E(偶数页)字母的组合。
%\fancyhf 用于同时定义页眉和页脚,习惯上使用 \fancyhf{} 来清空页眉页脚的设置。


% 在导言区使用此代码
%\usepackage{fancyhdr}
%\pagestyle{fancy}
%\renewcommand{\chaptermark}[1]{\markboth{#1}{}}
%\renewcommand{\sectionmark}[1]{\markright{\thesection\ #1}}
%\fancyhf{}
%\fancyfoot[C]{\bfseries\thepage}
%\fancyhead[LO]{\bfseries\rightmark}
%\fancyhead[RE]{\bfseries\leftmark}
%\renewcommand{\headrulewidth}{0.4pt} % 注意不用 \setlength
%\renewcommand{\footrulewidth}{0pt}

%fancyhdr还支持用\fancypagestyle 为自定义的页眉页脚样式命名,
%或者重新定义已有的样式如 plain 等:
% 自定义 myfancy 样式
%\fancypagestyle{myfancy}{%
%\fancyhf{}
%\fancyhead{...}
%\fancyfoot{...}
%}
% 使用样式
%\pagestyle{myfancy}








\end{document}