%!TEX program = xelatex

\documentclass[12pt]{ctexrep}
%4.1AMS宏集
\usepackage{amsmath}
\usepackage{amssymb}
\usepackage{bm}
\usepackage{amsthm}

\DeclareMathOperator{\argh}{argh}
\DeclareMathOperator*{\nut}{Nut}

\begin{document}
%缩进
%\noindent:消除首行缩进两字符。
%\setlength{\parindent}{2em}
%首行全部缩进两字节,该命令会影响其后的所有段落。
%换行
%\\:换行不缩进
%\par:换行+缩进两字符
%直接空一行:换行+缩进两字符
%空一行
%\\ \hspace*{\fill} \\:可以使两行之间空出一行的距离。

%4.2公式排版基础
%4.2.1行内和行间公式
\setlength{\parindent}{0em}
The Pythagorean throrem is $a^2 + b^2 = c^2$.

\hspace*{\fill} \\
The Pythagorean throrem is:
\begin{equation}
a^2 + b^ = c^2 \label{pythagorean}   
\end{equation}
Equation \eqref{pythagorean} is 
called `Gougu theorem' in Chinese.
%equation 环境为公式自动生成一个编号,
%这个编号可以用 \label 和 \ref 生成交叉引用,
%amsmath 的 \eqref 命令甚至为引用自动加上圆括号;
%还可以用 \tag 命令手动修改公式的编号,
%或者用 \notag 命令取消为公式编号
%(与之基本等效的命令是 \nonumber)

\hspace*{\fill} \\
It's wrong to say
\begin{equation}
1 + 1 = 3 \tag{dumb}
\end{equation}
or
\begin{equation}
1 + 1 = 4 \notag
\end{equation}

%TEX 原生排版行间公式的方法是用一对 $$ 符号包裹,
%不过无法通过指定 fleqn 选项控制左对齐,
%与上下文之间的间距也不好调整,故不太推荐使用
%使用不带编号的行间公式,则将公式用命令 \[ 和 \] 包裹1,
%与之等效的是displaymath 环境

\begin{equation*}
a^2 + b^2 = c^2    
\end{equation*}
For short:
\[ a^2 + b^2 = c^2 \]
Or if you like the long one:
\begin{displaymath}
a^2 + b^2 = c^2
\end{displaymath}

In text:
$\lim_{n \to \infty}
\sum_{k=1}^n \frac{1}{k^2}
= \frac{\pi^2}{6}$.

In display:
\[
\lim_{n \to \infty}
\sum_{k=1}^n \frac{1}{k^2}
= \frac{\pi^2}{6}    
\]

%行间公式的对齐、编号位置等性质由文档类选项控制,
%文档类的 fleqn 选项令行间公式左对齐;
%leqno 选项令编号放在公式左边。

%4.2.2数学模式
%数学模式相比于文本模式有以下特点:
%1. 数学模式中输入的空格被忽略。
%数学符号的间距默认由符号的性质(关系符号、运算符等)决定。
%需要人为引入间距时,使用 \quad 和 \qquad 等命令。
%2. 不允许有空行(分段)。
%行间公式中也无法用 \\ 命令手动换行。
%3. 所有的字母被当作数学公式中的变量处理,
%字母间距与文本模式不一致,也无法生成单词之间的空格。
%在数学公式中输入正体的文本,简单情况下可用\mathrm 命令。
%或者用 amsmath 提供的 \text 命令

%4.3数学符号
%4.3.1一般符号
$a_1, a_2, \dots, a_n$ \\
$a_1, a_2, + \cdots + a_n$
$\alpha, \beta, \Gamma, \Delta, \infty, \dots, \cdots, \vdots, \ddots$

%4.3.2指数、上下标和导数
$p^3_{ij} \qquad
m_\mathrm{Knuth}\qquad
\sum_{k=1}^3 k$\\[5pt]
$a^x+y \neq a^{x+y}\qquad
e^{x^2} \neq {e^x}^2$

$f(x) = x^2 \quad f'(x) = \quad f''^{2}(x) = 4$

%4.3.3分式和根式
%分式使用 \frac{分子}{分母} 
%amsmath 提供了方便的命令 \dfrac 和 \tfrac,
%令用户能够在行内使用正常大小的分式,或是反过来
In display style:
\[
3/8 \qquad \frac{3}{8} \qquad \tfrac{3}{8}    
\]
In text style:
$1\frac{1}{2}$~hours \qquad $1\dfrac{1}{2}$~hours

%一般的根式使用 \sqrt{...};
%表示 n 次方根时写成 \sqrt[n]{...}。
$\sqrt{x} \Leftrightarrow x^{1/2} \quad \sqrt[3]{2} \quad \sqrt{x^{2} + \sqrt{y}}$

%特殊的分式形式,
%如二项式结构,由 amsmath 宏包的 \binom 命令生成:
Pascal's rule is 
\[
\binom{n}{k} = \binom{n-1}{k} + \binom{n-1}{k-1}    
\]

%4.3.4关系符
\[
\ne, \ge, \le, \approx, \equiv, \propto, \sim, f_n(x) \stackrel{*}{\approx} 1    
\]
%倾斜的关系符号 ⩽ (\leqslant) 和 ⩾ (\geqslant) 由 amssymb 提供

%4.3.5算符
\[
\lim_{x \rightarrow 0}
\frac{\sin x}{x}=1    
\]

$a\bmod b \\ x\equiv a \pmod{b}$

%amsmath 允许用户在导言区用 \DeclareMathOperator
%定义自己的算符,其中带星号的命令定义带上下限的算符:
\[
\argh 3 = \nut_{x=1} 4x    
\]

%4.3.6巨算符
%积分号(\int)、求和号(\sum) 等符号称为巨算符。
%巨算符在行内公式和行间公式的大小和形状有区别
In text:
$\sum_{i=1}^n \quad 
\int_0^{\frac{\pi}{2}} \quad 
\oint_0^{\frac{\pi}{2}} \quad 
\prod_\epsilon $\\
In display:
\[
\sum_{i=1}^n \quad 
\int_0^{\frac{\pi}{2}} \quad 
\oint_0^{\frac{\pi}{2}} \quad 
\prod_\epsilon    
\]

%巨算符的上下标位置可由 \limits 和 \nolimits 调整,
%前者令巨算符类似 lim 或求和算符,上下标位于上下方;
%后者令巨算符类似积分号,上下标位于右上方和右下方。
In text:
$\sum\limits_{i=1}^n \quad 
\int\limits_0^{\frac{\pi}{2}} \quad 
\prod\limits_\epsilon $ \\
In display:
\[
\sum\nolimits_{i=1}^n \quad 
\int\limits_0^{\frac{\pi}{2}} \quad 
\prod\nolimits_\epsilon     
\]

%amsmath 宏包还提供了\substack,能够在下限位置书写多行表达式;
%subarray 环境更进一步,令多行表达式可选择居中 (c) 或左对齐 (l)
% \usepackage{amssymb}
\[
\sum_{\substack{0\le i\le n \\ j\in \mathbb{R}}} P(i,j) = Q(n)    
\]
\[
\sum_{\begin{subarray}{l}
    0\le i\le n \\
    j\in \mathbb{R}
\end{subarray}}
P(i,j) = Q(n)    
\]

%4.3.7数学重音和上下括号
$\bar{x_0} \quad \bar{x}_0$\\[5pt]
$\vec{x_0} \quad \vec{x}_0$\\[5pt]
$\hat{\mathbf{e}_x} \quad \hat{\mathbf{e}}_x$

$0.\overline{3} = \underline{\underline{1/3}}$\\[5pt]
$\hat{XY} \qquad \widehat{XY}$\\[5pt]
$\vec{AB} \qquad \overrightarrow{AB}$

$\underbrace{\overbrace{(a+b+c)}^6 
\cdot \overbrace{(d+e+f)}^7}
_\text{meaning of life} = 42$

%4.3.8箭头
%常用的箭头包括\rightarrow (→,或 \to)、\leftarrow(←,或\gets)等
%amsmath的\xleftarrow 和\xrightarrow 命令提供了长度可以伸展的箭头,
%并且可以为箭头增加上下标:
\[
a\xleftarrow{x+y+z} b     
\]
\[
c\xrightarrow[x<y]{a*b*c}d    
\]

%4.3.9括号和定界符
%LATEX 提供了多种括号和定界符表示公式块的边界,
%如小括号 ()、中括号 []、大括号 {}(\{\})、尖括号 ⟨⟩(\langle \rangle)等
${a,b,c} \neq \{a,b,c\}$
%使用 \left 和 \right 命令可令括号(定界符)的大小可变,
%在行间公式中常用。LATEX 会自动根据括号内的公式大小决定定界符大小。
%\left 和 \right 必须成对使用。
%需要使用单个定界符时,另一个定界符写成 \left. 或 \right.。
\[
1 + \left(\frac{1}{1-x^{2}}
\right)^3 \qquad
\left.\frac{\partial f}{\partial t}
\right|_{t=0}    
\]

%用\big、\bigg等命令生成固定大小的定界符。
%更常用的形式是类似 \left 的 \bigl、\biggl 等,
%以及类似\right的\bigr、\biggr等(\bigl 和\bigr 不必成对出现)。
$\Bigl((x+1)(x-1)\Bigr)^{2}$\\
$\bigl( \Bigl( \biggl( \Biggl( \quad 
\bigr\} \Bigr\} \biggr\} \Biggr\} \quad 
\big\| \Big\| \bigg\| \Bigg\| \quad 
\big\Downarrow \Big\Downarrow 
\bigg\Downarrow \Bigg\Downarrow$

%使用 \big 和 \bigg 等命令的另外一个好处是:
%用 \left 和 \right 分界符包裹的公式块是不允许断行的
%(下文提到的 array 或者 aligned 等环境视为一个公式块),
%所以也不允许在多行公式里跨行使用,
%而 \big 和 \bigg 等命令不受限制。

%4.4多行公式
%4.4.1长公式折行
%通常来讲应当避免写出超过一行而需要折行的长公式。
%如果一定要折行的话,习惯上优先在等号之前折行,
%其次在加号、减号之前,再次在乘号、除号之前。
%其它位置应当避免折行。
%amsmath 宏包的 multline 环境提供了书写折行长公式的方便环境。
%它允许用 \\ 折行,将公式编号放在最后一行。
%多行公式的首行左对齐,末行右对齐,其余行居中。
\begin{multline}
a + b + c + d + e + f
+ g + h + i \\
= j + k + l + m + n\\
= o + p + q + r + s\\
= t + u + v + x + z  
\end{multline}
%类似 equation*,multline* 环境排版不带编号的折行长公式。

%4.4.2多行公式
%目前最常用的是align 环境,它将公式用 & 隔为两部分并对齐。
%分隔符通常放在等号左边
\begin{align}
a & = b + c \\
& = d + e
\end{align}

%align 环境会给每行公式都编号。
%我们仍然可以用 \notag 去掉某行的编号。
%为了对齐等号,我们将分隔符放在右侧,
%并且此时需要在等号后添加一对括号 {} 以产生正常的间距:
\begin{align}
a ={} & b + c \\
  ={} & d + e + f + g + h + i
  + j + k + l \notag \\
  & + m + n +o \\
  ={} & p + q + r +s 
\end{align}

%align 还能够对齐多组公式,
%除等号前的 & 之外,公式之间也用 & 分隔
\begin{align}
a &=1  & b &=2  & c &=3 \\
d &=-1 & e &=-2 & f &=-5
\end{align}

%不需要按等号对齐,只需罗列数个公式,
%gather 将是一个很好用的环境

\begin{gather}
a = b + c \\
d = e + f + g \\
h + i = j + k \notag \\
l + m = n
\end{gather}
%align 和 gather 有对应的不带编号的版本 align* 和 gather*。

%4.4.3公用编号的多行公式
\begin{equation}
\begin{aligned}
a &= b + c \\
d &= e + f +g \\
h + i &= j + k \\
l + m &= n
\end{aligned}
\end{equation}

%split 环境和 aligned 环境用法类似,
%也用于和 equation 环境套用,
%区别是 split 只能将每行的一个公式分两栏,
%aligned 允许每行多个公式多栏。

%4.5数组和矩阵
\[
\mathbf{X} = \left( 
\begin{array}{cccc}
x_{11} & x_{12} & \ldots & x_{1n}\\
x_{21} & x_{22} & \ldots & x_{2n}\\
\vdots & \vdots  & \ldots & \vdots\\
x_{n1} & x_{n2} & \ldots & x_{nn}\\  
\end{array}  \right)
\]

%利用空的定界符
\[
|x| = \left\{ 
\begin{array}{rl}
-x & \text{if} x<0,\\
0  & \text{if} x=0,\\
x  & \text{if} x>0.
\end{array} \right.
\]

%amsmath 提供的 cases 环境
\[
|x| = 
\begin{cases}
-x & \text{if } x<0,\\
0 & \text{if } x=0,\\
x & \text{if } x>0.
\end{cases}  
\]

%amsmath 宏包还直接提供了多种排版矩阵的环境,
%包括不带定界符的 matrix,以及带各种定界符的矩阵 
%pmatrix(()、bmatrix([)、Bmatrix({)、vmatrix(|)、Vmatrix(||)。
%使用这些环境时,无需给定列格式

\[
\begin{matrix}
1 & 2 \\ 3 & 4
\end{matrix}  \qquad
\begin{bmatrix}
x_{11} & x_{12} & \ldots & x_{1n}\\
x_{21} & x_{22} & \ldots & x_{2n}\\
\vdots & \vdots & \ddots & \vdots\\
x_{n1} & x_{n2} & \ldots & x_{nn}\\
\end{bmatrix}
\]

%在矩阵中的元素里排版分式时,
%一来要用到 \dfrac 等命令,
%二来行与行之间有可能紧贴着,
%这时要用到 3.6.6 小节的方法来调节间距:
\[
\mathbf{H}=
\begin{bmatrix}
\dfrac{\partial^2 f}{\partial x^2} &
\dfrac{\partial^2 f}{\partial x \partial y} \\[8pt]
\dfrac{\partial^2 f}{\partial x \partial y} &
\dfrac{\partial^2 f}{\partial y^2}
\end{bmatrix} 
\]

%4.6公式中的间距
%在公式中我们还可能用到的间距包括 
%\,、\:、\; 以及负间距 \!,
%其中 \quad 、\qquad 和 \, 在文本和数学环境中可用,
%后三个命令只用于数学环境。文本中的\ 也能使用在数学公式中。

%修正积分的被积函数 f(x) 和微元 dx 之间的距离
\[
\int_a^b f(x)\mathrm{d}x
\qquad
\int_a^b f(x)\,\mathrm{d}x  
\]

%另一个用途是生成多重积分号
%如果我们直接连写两个 \int,之间的间距将会过宽,
%此时可以使用负间距 \! 修正之。
%不过 amsmath 提供了更方便的多重积分号,
%如二重积分 \iint、三重积分 \iiint 等。
\newcommand\diff{\,\mathrm{d}}
\begin{gather*}
\int\int f(x)f(y)
\diff x \diff y \\
\int\!\!\!\int
f(x)f(y) \diff x \diff y \\
\iint f(x)f(y) \diff x \diff y \\
\iint\quad \iiint\quad \idotsint
\end{gather*}

%4.7数学符号的字体控制
%4.7.1数学字母字体
% \usepackage{amssymb}
$\mathcal{R} \quad \mathfrak{R}
\quad \mathbb{R}$
\[
\mathcal{L}
= -\frac{1}{4}F_{\mu\mu}F^{\mu\mu}  
\]
$\mathfrak{su}(2)$ and $\mathfrak{so}(3)$ Lie algebra

%一般来说,不同的数学字体往往带有不同的语义,
%如矩阵、向量等常会使用粗体或粗斜体,
%而数集常会使用 \mathbb 表示。

%4.7.2
%\mathbf 命令只能获得直立、加粗的字母。
%如果想得到粗斜体6,可以使用 amsmath 宏包提供的 
%\boldsymbol 命令
$\mu, M \qquad \boldsymbol{\mu}, \boldsymbol{M}$
%也可以使用 bm 宏包提供的 \bm 命令:
% \usepackage{bm}
$\mu, M \qquad \bm{\mu}, \bm{M}$
%在 LATEX 默认的数学字体中,一些符号本身并没有粗体版本,
%使用 \boldsymbol 也得不到粗体。此时 \bm 命令会生成“伪粗体”,
%尽管效果比较粗糙,但在某些时候也不失为一种解决方案。


%4.7.3数学符号的尺寸
%\displaystyle行间公式尺寸
%\textstyle行内公式尺寸
%\scriptstyle上下标尺寸
%\scriptscriptstyle次级上下标尺寸
\[
r = \frac{\sum_{i=1}^n (x_i- x)(y_i- y)}
{\displaystyle \left[
  \sum_{i=1}^n (x_i-x)^2
  \sum_{i=1}^n (y_i-y)^2
\right]^{1/2}}  
\]

%4.8定理环境
%4.8.1LATEX原始的定理环境
%\newtheorem{⟨theorem environment⟩}{⟨title⟩}[⟨section-level⟩]
%\newtheorem{⟨theorem environment⟩}[⟨counter⟩]{⟨title⟩}
\newtheorem{mythm}{My Theorem}[section]
\begin{mythm}\label{thm:light}
The light speed in vacuum
is $299,792,458\,\mathrm{m/s}$.
\end{mythm}
\begin{mythm}[Energy-momentum relation]
The relationship of energy,
momentum and mass is 
\[
E^2 = m_0^2 c^4 + p^2 c^2\]
where $c$ is the light speed
described in theorem \ref{thm:light}.
\end{mythm}

%4.8.2amsthm宏包
%amsthm 提供了 \theoremstyle 命令支持定理格式的切换,
%在用 \newtheorem 命令定义定理环境之前使用。
%amsthm 预定义了三种格式用于 \theoremstyle:
%plain 和 LATEX 原始的格式一致;
%definition 使用粗体标签、正体内容;
%remark 使用斜体标签、正体内容
%amsthm 还支持用带星号的 \newtheorem* 定义不带序号的定理环境
\theoremstyle{definition} \newtheorem{law}{Law}
\theoremstyle{plain} \newtheorem{jury}[law]{Jury}
\theoremstyle{remark} \newtheorem*{mar}{Margaret}

\begin{law}\label{law:box}
Don't hide in the witness box.
\end{law}
\begin{jury}[The Twelve]
It could be you! So beware and
see law~\ref{law:box}.\end{jury}
\begin{jury}
You will disregard the last 
statement. \end{jury}
\begin{mar}No,No,No\end{mar}
\begin{mar}Denis!\end{mar}

%amsthm 还支持使用 \newtheoremstyle 命令自定义定理格式,
%更为方便使用的是 ntheorem宏包。

%4.8.3证明环境和证毕符号
%amsthm 还提供了一个 proof 环境用于排版定理的证明过程。
%proof 环境末尾自动加上一个证毕符号:

\begin{proof}
For simplicity, we use 
\[
E=mc^2  
\]
That's it.
\end{proof}

%如果行末是一个不带编号的公式,符号会另起一行,
%这时可使用 \qedhere 命令将符号放在公式末尾:
\begin{proof}
For simplicity, we use 
\[
E=mc^2 \qedhere  
\]
\end{proof}

%\qedhere 对于 align* 等环境也有效:
\begin{proof}
Assuming $\gamma = 1/\sqrt{1-v^2/c^2}$, then
\begin{align*}
E &= \gamma m_0 c^2 \\
p &= \gamma m_0v \qedhere
\end{align*}
\end{proof}

%在使用带编号的公式时,建议最好不要在公式末尾使用\qedhere 命令。
%对带编号的公式使用\qedhere 命令会使符号放在一个难看的位置,紧贴着公式
\begin{proof}
For simplicity, we use 
\begin{equation}
E=mc^2.\qedhere
\end{equation}
\end{proof}

%在 align 等环境中使用 \qedhere 命令会使盖掉公式的编号;
%使用 equation 嵌套aligned 等环境时,
%\qedhere 命令会将直接放在公式后。这些位置都不太正常。

%证毕符号本身被定义在命令 \qedsymbol 中,
%如果有使用实心符号作为证毕符号的需求,
%需要自行用\renewcommand 命令修改
\renewcommand{\qedsymbol}{\rule{1ex}{1.5ex}}
\begin{proof}
For simplicity, we use
\[
E=mc^2 \qedhere  
\]
\end{proof}

%4.9符号表
%4.9.1LATEX普通符号















\end{document}