%!TEX program = xelatex

\documentclass[10pt]{ctexart}

\usepackage{xparse}

\begin{document}
%8.1自定义LATEX命令和功能
%8.1.1自定义新命令
%使用如下命令可以定义你自己的命令:
%\newcommand{\⟨name⟩}[⟨num⟩]{⟨definition⟩}
%\newcommand 的基本用法需要两个必选参数,
%第一个参数 ⟨name⟩ 是要定义的命令名称(带反斜线),
%第二个参数 ⟨definition⟩ 是命令的具体定义。
%方括号里的参数 ⟨num⟩ 是可选的,
%用于指定新命令所需的参数数目(最多9个)。
%如果缺省可选参数,默认就是 0,也就是新定义的命令不带任何参数。

\newcommand{\tnss}{The not so Short
Introduction to \LaTeXe}
This is ``\tnss'' \ldots{} ``\tnss''

\newcommand{\txsit}[1]{This is the 
\emph{#1} Short Introduction to \LaTeXe}
% in the document body:
\begin{itemize}
      \item \txsit{not so}
      \item \txsit{very}
\end{itemize}

%LATEX 不允许使用 \newcommand 定义一个与现有命令重名的命令。
%如果需要修改命令定义的话,使用 \renewcommand 命令。
%它使用与命令 \newcommand 相同的语法。
%在某些情况之下,使用 \providecommand 命令是一种比较理想的方案:
%在命令未定义时,它相当于 \newcommand;在命令已定义时,
%沿用已有的定义。

%8.1.2定义环境
%与\newcommand 命令类似,可以用\newenvironment 定义新的环境。
%它的语法如下所示:
%\newenvironment{⟨name⟩}[⟨num⟩]{⟨before⟩}{⟨after⟩}
%同样地,\newenvironment 命令有一个可选的参数。
%在 ⟨before⟩ 中的内容将在此环境包含的文本之前处理,
%而在 ⟨after⟩ 中的内容将在遇到 \end{⟨name⟩} 命令时处理。
%下面的例子演示了 \newenvironment 命令的用法:

\newenvironment{king}
{\rule{1ex}{1ex}%
\hspace{\stretch{1}}}
{\hspace{\stretch{1}}%
\rule{1ex}{1ex}}

\begin{king}
My humble subjects \ldots  
\end{king}

%你确实希望改变一个现有的环境,
%你可以使用命令 \renewenvironment,
%它使用和命令 \newenvironment 相同的语法。

%8.1.3xparse宏包简介
%如果需要定义带有多个可选参数、或者带星号的命令或环境,
%可以使用 xparse 宏包。
%它提供了\NewDocumentCommand和\NewDocumentEnvironment等命令,
%具体语法如下:
%\NewDocumentCommand\⟨name⟩{⟨arg spec⟩}{⟨definition⟩}
%\NewDocumentEnvironment{⟨name⟩}{⟨arg spec⟩}{⟨before⟩}{⟨after⟩}

%-NoValue- 标记可以用 \IfNoValueTF 等命令来判断:
%\IfNoValueTF{⟨argument⟩}{⟨true code⟩}{⟨false code⟩}
%\IfNoValueT{⟨argument⟩}{⟨true code⟩}
%\IfNoValueF{⟨argument⟩}{⟨false code⟩}

% 百分号用于注释掉不必要的空格和换行符
\NewDocumentCommand\hello{om}
{%
\IfNoValueTF{#1}%
{Hello, #2!}%
{Hello, #1 and #2!}%
}
\hello{Alice}
\hello[Bob]{Alice}

%\BooleanTrue 和 \BooleanFalse 
%则可以用 \IfBooleanTF 等命令来判断:
%\IfBooleanTF{⟨argument⟩}{⟨true code⟩}{⟨false code⟩}
%\IfBooleanT{⟨argument⟩}{⟨true code⟩}
%\IfBooleanF{⟨argument⟩}{⟨false code⟩}

\NewDocumentCommand\hereis{sm}
{Here is \IfBooleanTF{#1}{an}{a} #2.}
\hereis{banana}
\hereis*{apple}

%需要注意的是,与命令不同,环境在定义时名字里面可以包含 *:
%\NewDocumentEnvironment {mytabular}{ o +m } {...} {...}
%\NewDocumentEnvironment {mytabular*} { m o +m } {...} {...}
%用 s 标记的 * 则应该放在 \begin{⟨env⟩} 的后面:
%\NewDocumentEnvironment { envstar } { s }
%{\IfBooleanTF {#1} {star} {no star}} {}
%\begin{envstar}*
%\end{envstar}

%8.2编写自己的宏包和文档类
%8.2.1编写简单的宏包
%如果定义了很多新的环境和命令,文档的导言区将变得很长,
%在这种情况下,可以建立一个新的 LATEX 宏包来存放所有
%你自己定义的命令和环境,
%然后在文档中使用 \usepackage 命令来调用自定义的宏包。
%写一个宏包的基本工作就是将原本在你的文档导言区里很长的内容
%拷贝到另一个文件中去,
%这个文件需要以 .sty 作扩展名。你还需要加入一个宏包专用的命令:
%\ProvidesPackage{⟨package name⟩}

% Demo Package by Tobias Oetiker
%\ProvidesPackage{demopack}
%\newcommand{\tnss}{The not so Short Introduction
%to \LaTeXe}
%\newcommand{\txsit}[1]{The \emph{#1} Short
%Introduction to \LaTeXe}
%\newenvironment{king}{\begin{quote}}{\end{quote}}

%8.2.2在宏包中调用其他宏包
%如果你想进一步把各种宏包的功能汇总到一个文件里,
%而不是在文档的导言区罗列一大堆宏包的话,
%LATEX 允许你在自己编写的宏包中调用其它宏包,
%命令为 \RequirePackage,用法和\usepackage 一致:
%\RequirePackage[⟨options⟩]{⟨package name⟩}

%8.2.3编写自己的文档类
%自己的文档类以 .cls 作扩展名,开头使用 \ProvidesClass 命令
%\ProvidesClass{⟨class name⟩}
%在你的文档类中调用其它文档类的命令是 \LoadClass,
%用法和 \documentclass 十分相像
%\LoadClass[⟨options⟩]{⟨package name⟩}

%8.3计数器
%8.3.1定义和修改计数器
%定义一个计数器的方法为:
%\newcounter{⟨counter name⟩}[⟨parent counter name⟩]
%⟨counter name⟩ 为计数器的名称。计数器可以有上下级的关系,
%可选参数 ⟨parent countername⟩ 定义为 ⟨counter name⟩ 
%的上级计数器。
%以下命令修改计数器的数值,\setcounter 将数值设为 ⟨number⟩;
%\addtocounter 将数值加上 ⟨number⟩;\stepcounter 将数值加一,
%并将所有下级计数器归零。
%\setcounter{⟨counter name⟩}{⟨number⟩}
%\addtocounter{⟨counter name⟩}{⟨number⟩}
%\stepcounter{⟨counter name⟩}

%8.3.2计数器的输出格式
%计数器 ⟨counter⟩ 的输出格式由 \the⟨counter⟩ 表示,
%如我们在 5.5 一节见过的 \thechapter等。
%这个值默认以阿拉伯数字形式输出,如果想改成其它形式,
%需要重定义 \the⟨counter⟩,
%如将 equation 计数器的格式定义为大写字母:
%\renewcommand\theequation{\Alph{equation}}
%命令 \Alph 控制计数器 ⟨counter⟩ 的值以大写字母形式显示。

%计数器的输出格式还可以利用其它字符,
%甚至其它计数器的输出格式与之组合。
%如标准文档类里对\subsection相关的计数器的输出格式的定义相当于:
%\renewcommand\thesubsection{\thesection.\arabic{subsection}}

%8.3.3LATEX中的计数器
%• 所有章节命令 \chapter、\section 等分别对应计数器 
%chapter、section 等等,而且有上下级的关系。
%而计数器 part 是独立的。
%• 有序列表 enumerate 的各级计数器为 
%enumi, enumii, enumiii, enumiv,也有上下级的关系。
%• 图表浮动体的计数器就是 table 和 figure;公式的计数器为
%equation。这些计数器在 article文档类中是独立的,
%而在 report 和 book 中以 chapter 为上级计数器。
%• 页码、脚注的计数器分别是 page 和 footnote。

%我们可以利用前面介绍过的命令,
%修改计数器的样式以达到想要的效果,比如把页码修改成大写罗马数字,
%左右加横线,或是给脚注加上方括号:
%\renewcommand\thepage{--~\Roman{page}~--}
%\renewcommand\thefootnote{[\arabic{footnote}]}

%secnumdepth、tocdepth

%8.4LATEX可定制的一些命令和参数






\end{document}