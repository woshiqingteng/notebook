%!TEX program = xelatex

\documentclass[10pt]{ctexart}

\usepackage{makeidx}
\usepackage{xcolor}
\usepackage{hyperref}

\makeindex


\bibliographystyle{plain}


\begin{document}

%6.1参考文献和BIBTEX工具
%6.1.1基本的参考文献和引用
%\cite{⟨citation⟩}
%⟨citation⟩ 为引用的参考文献的标签,
%类似 \ref 里的参数;\cite 带一个可选参数,
%为引用的编号后加上额外的内容,
%如\cite[page 22]{Paper2013}
%可能得到形如 [13, page 22] 这样的引用。
%参考文献由 thebibliography 环境包裹。
%每条参考文献由 \bibitem 开头,其后是参考文献本身的内容:
%\begin{thebibliography}{⟨widest label⟩}
%\bibitem[⟨item number⟩]{⟨citation⟩} ...
%\end{thebibliography}
%其中 ⟨citation⟩ 是 \cite 使用的文献标签,
%⟨item number⟩ 自定义参考文献的序号,
%如果省略,则按自然排序给定序号。
%⟨widest label⟩ 用以限制参考文献序号的宽度,
%如 99 意味着不超过两位数字。
%通常设定为与参考文献的数目一致。

%thebibliography 环境自动生成不带编号的一节(article 文档类)
%或一章(report / book 文档类)。
%在 article 文档类的节标题默认为“References”,
%而在 report / book 文档类的章标题默认为“Bibliography”。
%用户可通过 8.4 节给出的方法定制参考文献的标题。


\section{Introduction}
Part1~\cite{germenTeX} has proposed that \ldots

\begin{thebibliography}{99}
\bibitem{germenTeX} H.~Part1: \emph{German \TeX},
TUGboat Volume~9, Issue~1(1988)
\end{thebibliography}


%6.1.2BIBTEX数据库

%6.1.3BIBTEX样式
%使用样式文件的方法是在源代码内(一般在导言区)
%使用 \bibliographystyle 命令:
%\bibliographystyle{⟨bst-name⟩}

\section{Some words}
Some excellent books, for example, \cite{Alice13}
and \cite{Alice13} \ldots

\bibliography{books}

%\bibliographystyle 和 \bibliography 命令缺一不可
%xelatex demo
%bibtex demo
%xelatex demo
%xelatex demo

%6.1.5natbib宏包
%除了 \cite 之外,natbib 宏包在正文中支持两种引用方式:
%\citep{⟨citation⟩}
%\citet{⟨citation⟩}

%natbib 宏包同样也支持数字引用,并且支持将引用的序号压缩,例如:
%\usepackage[numbers,sort&compress]{natbib}
%调用 natbib 宏包时指定以上选项后,连续引用多篇文献时,
%会生成形如 (3–7) 的引用而不是 (3,4, 5, 6, 7)。

%6.1.6biblatex宏包
%biblatex 宏包也因其对 UTF-8 和中文参考文献的良好支持,
%被国内较多 LATEX 模板采用。


\vspace{2ex}
\noindent
文档结构和 biblatex 相关命令\\
1.首先是在导言区调用 biblatex 宏包。
宏包支持以 ⟨key⟩=⟨value⟩ 形式指定选项,
包括参考文献样式 style、参考文献著录排序的规则 sorting 等。\\
2.接着在导言区使用\verb|\addbibresource|命令为
biblatex引入参考文献数据库。与基于 BIBTEX
的传统方式不同的是,这里需要写完整的文件名。\\
3.在正文中使用\verb|\cite| 命令引用参考文献。
除此之外还可以使用丰富的命令达到不同的引用效果,
如\verb|\citeauthor|和\verb|\citeyear|
分别单独引用作者和年份,\verb|\textcite|和\verb|\parencite|
分别类似natbib宏包提供的\verb|\citet|和\verb|\citep| 命令,
以及脚注式引用\verb|\footcite|等。\\
4.最后在需要排版参考文献的位置使用命令
\verb|\printbibliography|。

\vspace{2ex}
\noindent
编译方式\\
与基于 BIBTEX 的传统方式不同的是,
biblatex 宏包使用 biber 程序处理参考文献。
因此上述文档的编译步骤为:\\
xelatex demo\\
biber demo\\
xelatex demo\\
xelatex demo

\vspace{2ex}
\noindent

\begin{verbatim}
% File: egbibdata.bib
@book{caimin2006,
    title = {UML 基础和 Rose 建模教程},
    address = {北京},
    author = {蔡敏 and 徐慧慧 and 黄柄强},
    publisher = {人民邮电出版社},
    year = {2006},
    month = {1}
}
\end{verbatim}

\vspace{2ex}
\begin{verbatim}
% File: demo.tex
\documentclass{ctexart}
% 使用符合 GB/T 7714-2015 规范的参考文献样式
\usepackage[style=gb7714-2015]{biblatex}
% 注意加 .bib 扩展名
\addbibresource{egbibdata.bib}

\begin{document}

见文献\cite{caimin2006}。

\printbibliography
\end{document}
\end{verbatim}

\noindent
biblatex 样式和其它选项\\
biblatex 使用的参考文献样式分为著录样式
(bibliography style)和引用样式(citation style),
分别以 .bbx 和 .cbx 为扩展名。
参考文献的样式在调用宏包时使用 style 选项指定,
或者使用bibstyle 或 citestyle 分别指定:\\
\% 同时调用 gb7714-2015.bbx 和 gb7714-2015.cbx
\begin{verbatim}
\usepackage[style=gb7714-2015]{biblatex}    
\end{verbatim}
\% 著录样式调用 gb7714-2015.bbx,引用样式调用 biblatex 宏包自带的 authoryear
\begin{verbatim}
\usepackage[bibstyle=gb7714-2015,citestyle=authoryear]{biblatex}  
\end{verbatim}

%6.2索引和makeindex工具
%6.2.1使用makeindex工具的方法
\noindent
第一步,在 LATEX 源代码的导言区调用 makeidx 宏包,
并使用\verb|\|makeindex 命令开启索引的收集:\\
\verb|\usepackage{makeidx}|\\
\verb|\makeindex| \\
第二步,在正文中需要索引的地方使用 \verb|\|index 命令。
\verb|\|index 命令的参数写法详见下一小节;
并在需要输出索引的地方(如所有章节之后)
使用\verb|\|printindex 命令。\\
第三步,编译过程:\\
1. 首先用 xelatex 等命令编译源代码 demo.tex。
编译过程中产生索引记录文件 demo.idx;\\
2. 用 makeindex 程序处理 demo.idx,
生成用于排版的索引列表文件 demo.ind;\\
3. 再次编译源代码 demo.tex,正确生成索引列表。

%6.2.2索引项的写法
\noindent
添加索引项的命令为:\\
\verb|\index{⟨index entry⟩}|\\
其中 ⟨index entry⟩ 为索引项
\newpage
Test index.
\index{Test@\textsf{""Test}|(textbf}
\index{Text@\textsf{""Test}!sub@"|sub"||see{Test}}
\newpage
Test index.
\index{Test@\textsf{""Test}|)textbf}

\printindex

%6.3使用颜色
原始的 LATEX 不支持使用各种颜色。
color 宏包或者 xcolor 宏包提供了对颜色的支持,给
PDF 输出生成颜色的特殊指令。\\
%6.3.1颜色的表达方式
调用 color 或 xcolor 宏包后,我们就可以用如下命令切换颜色:\\
\verb|\color[⟨color-mode⟩]{⟨code⟩}|\\
\verb|\color{⟨color-name⟩}|\\
颜色的表达方式有两种,其一是使用色彩模型和色彩代码,
代码用 0 \~{} 1 的数字代表成分的比例。
color 宏包支持 rgb、cmyk 和 gray 模型,
xcolor 支持更多的模型如 hsb 等。

\large\sffamily
{\color[gray]{0.6}
60\% 灰色} \\
{\color[rgb]{0,1,1}
青色}

\large\sffamily
{\color{red} 红色} \\
{\color{blue} 蓝色}

xcolor 还支持将颜色通过表达式混合或互补:\\
\large\sffamily
{\color{red!40} 40\% 红色}\\
{\color{blue}蓝色
\color{blue!50!black}蓝黑
\color{black}黑色}\\
{\color{-red}红色的互补色}

%我们还可以通过命令自定义颜色名称,
%注意这里的 ⟨color-mode⟩ 是必选参数:
%\definecolor{⟨color-name⟩}{⟨color-mode⟩}{⟨code⟩}

%6.3.2带颜色的文本和盒子
%color / xcolor 宏包都定义了一些方便用户使用的带颜色元素。

%输入带颜色的文本可以用类似 \textbf 的命令:
%\textcolor[⟨color-mode⟩]{⟨code⟩}{⟨text⟩}
%\textcolor{⟨color-name⟩}{⟨text⟩}

%以下命令构造一个带背景色的盒子,⟨material⟩ 为盒子中的内容:
%\colorbox[⟨color-mode⟩]{⟨code⟩}{⟨material⟩}
%\colorbox{⟨color-name⟩}{⟨material⟩}

%以下命令构造一个带背景色和有色边框的盒子,
%⟨fcode⟩ 或 ⟨fcolor-name⟩ 用于设置边框颜色:
%\fcolorbox[⟨color-mode⟩]{⟨fcode⟩}{⟨code⟩}{⟨material⟩}
%\fcolorbox{⟨fcolor-name⟩}{⟨color-name⟩}{⟨material⟩}

\sffamily
文字用\textcolor{red}{红色}强调\\
\colorbox[gray]{0.95}{浅灰色背景}\\
\fcolorbox{blue}{yellow}{%
\textcolor{blue}{蓝色边框+文字,黄色背景}
}

%6.4使用超链接
%6.4.1hyperref宏包
%hyperref 宏包提供了命令 \hypersetup 配置各种参数。
%参数也可以作为宏包选项,在调用宏包时指定:
%\hypersetup{⟨option1⟩,⟨option2⟩={value},…}
%\usepackage[⟨option1⟩,⟨option2⟩={value},…]{hyperref}
%当选项值为 true 时,可以省略“=true”不写。

%6.4.2超链接
%hyperref 宏包提供了直接书写超链接的命令,
%用于在 PDF 中生成 URL:
%\url{⟨url⟩}
%\nolinkurl{⟨url⟩}
%我们也可以像 HTML 中的超链接一样,把一段文字作为超链接:
%\href{⟨url⟩}{⟨text⟩}

\url{https://wikipedia.org} \\
\nolinkurl{https://wikipedia.org} \\
\href{https://wikipedia.org}{Wiki}

%使用 hyperref 宏包后,
%文档中所有的引用、参考文献、索引等等都转换为超链接。
%用户也可对某个 \label 命令定义的标签 ⟨label⟩ 作超链接
%(注意这里的 ⟨label⟩ 虽然是可选参数的形式,但通常是必填的
%\hyperref[⟨label⟩]{⟨text⟩}

%默认的超链接在文字外边加上一个带颜色的边框
%(在打印 PDF 时边框不会打印),
%可指定colorlinks 参数修改为将文字本身加上颜色,
%或修改 pdfborder 参数调整边框宽度以“去掉”边框;
%hidelinks 参数则令超链接既不变色也不加边框。
%\hypersetup{hidelinks}
% or:
%\hypersetup{pdfborder={0 0 0}}


%6.4.3PDF书签
%hyperref 宏包另一个强大的功能是为 PDF 生成书签。
%对于章节命令 \chapter、\section等,
%默认情况下会为 PDF 自动生成书签。
%和交叉引用、索引等类似,生成书签也需要多次编译源代码,
%第一次编译将书签记录写入 .out 文件,第二次编译才正确生成书签。

%hyperref 还提供了手动生成书签的命令:
%\pdfbookmark[⟨level⟩]{⟨bookmark⟩}{⟨anchor⟩}
%⟨bookmark⟩ 为书签名称,⟨anchor⟩ 为书签项使用的锚点
%(类似交叉引用的标签)。可选参数 ⟨level⟩为书签的层级,默认为 0。

%\texorpdfstring{⟨LATEX code⟩}{⟨PDF bookmark text⟩}
%比如在章节名称里使用公式 E = mc2,
%而书签则使用纯文本形式的 E=mc^2:
%\section{质能公式 \texorpdfstring{$E=mc^2$}{E=mc\textasciicircum 2}}

%6.4.4PDF文档属性









\end{document}