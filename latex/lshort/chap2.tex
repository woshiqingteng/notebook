%!TEX program = xelatex

%2.1
%计算机的基本存储单位是字节(byte),
%每个字节为八位(8-bit),
%范围用十六进制写作0x00–0xFF。
%ASCII(美国通用信息交换码)使用 0x00–0x7F 对文字编码,
%也就是 7-bit,覆盖了基本的拉丁字母、数字和符号,
%以及一些不可打印的控制字符(如换行符、制表符等)。

%2.2排版中文
\documentclass{ctexart}
\begin{document}
在\LaTeX{}中排版中文。
汉字和English单词混排,通常不需要在中英文之间添加额外的空格。
当然,为了代码的可读性,加上汉字和 English 之间的空格也无妨。
汉字换行时不会引入多余的空格。

%2.3LATEX中的字符
%2.3.1空格和分段
Several space    equal one.
    Front space are ignored.

An empty line starts a new paragrah.\par
A \verb|\par| command does the same.

%2.3.2
This is an % short comment
% ---
% Long and organized
% comments
% ---
example: Comments do not bre%
ak a word.

%2.3.3特殊字符
\# \$ \% \& \{ \} \_
\^{} \~{} \textbackslash

%2.3.4连字
It's difficult to find \ldots\\
It's dif{}f{}icult to f{}ind \ldots
%常见的连字(ligatures),ff/fi/fl/ffi/ffl

%2.3.5标点符号
%引号
``Please press the `x' key.''
%连字号和破折号
%连字号(hyphen)组成复合词、
%短破折号(en-dash)连接数字表示范围、
%长破折号(em-dash)连接单词

daughter-in-law, X-rated\\
pages 13--67\\
yes---or no?

%省略号
one, two, there, \ldots one hundred.
%波浪号

%2.3.6拉丁文扩展与重音
H\^otal, na\"\i ve, \'el\`eve,\\
sm\o rreber\o d, !`Se\ norital!,\\
Sch\"onbrunner Schlo\ss{}
Stra\ss e

%2.3.7其他符号
\P{} \S{} \dag{} \ddag{}
\copyright{} \pounds{}

\textasteriskcentered
\textperiodcentered
\textbullet

\textregistered{} \texttrademark

%LATEX标志

\TeX\\
\LaTeX\\
\LaTeXe

%2.4断行和断页

%2.4.1单词间距
Fig.~2a \\
Donald~E. Knuth

%2.4.2手动断行和断页
%\\[length]   \\*[length]
%可选参数length,在断行处向下增加垂直间距
%\\也在表格、公式等地方用于换行
%带星号的\\表示禁止在断行处分页
%\newline
%不带可选参数
%\newline只用于文本段落

另外需要注意的是,使用\verb|\\|断行处命令\\
不会令内容另起一段,而是在段落中直接开始新的一行。

%断页命令
%\newpage
%\clearpage
%双栏排版模式中\newpage起到另起一栏作用,\clearpage另起一页
%设计浮动体排版不同

%断页
%\linebreak[n] \nolinebreak[n]
%\pagebreak[n] \nopagebreak[n]
%n代表适合/不适合的程度,取值范围0-4,缺省为4

使用\verb|\newline| 断行的效果
\newline
与使用\verb|\linebreak|断行的效果
\linebreak
进行对比

%2.4.3断词
I think this is: su\-per\-cal\-%
i\-frag\-i\-lis\-tic\-ex\-pi\-%
al\-i\-do\-cious

\end{document}



